\documentclass[a4paper,10pt,twocolumn,oneside]{article}
\setlength{\columnsep}{10pt}                                                                    %兩欄模式的間距
\setlength{\columnseprule}{0pt}                                                                %兩欄模式間格線粗細

\usepackage{amsthm}								%定義,例題
\usepackage{amssymb}
%\usepackage[margin=2cm]{geometry}
\usepackage{fontspec}								%設定字體
\usepackage{color}
\usepackage[x11names]{xcolor}
\usepackage{listings}								%顯示code用的
%\usepackage[Glenn]{fncychap}						%排版,頁面模板
\usepackage{fancyhdr}								%設定頁首頁尾
\usepackage{graphicx}								%Graphic
\usepackage{enumerate}
\usepackage{multicol}
\usepackage{titlesec}
\usepackage{amsmath}
\usepackage[CheckSingle, CJKmath]{xeCJK}
\usepackage{savetrees}
% \usepackage{CJKulem}

%\usepackage[T1]{fontenc}
\usepackage{amsmath, courier, listings, fancyhdr, graphicx}
\topmargin=0pt
\headsep=5pt
\textheight=780pt
\footskip=0pt
\voffset=-40pt
\textwidth=545pt
\marginparsep=0pt
\marginparwidth=0pt
\marginparpush=0pt
\oddsidemargin=0pt
\evensidemargin=0pt
\hoffset=-42pt

\titlespacing\section{0pt}{0pt plus 2pt minus 2pt}{0pt plus 2pt minus 2pt}
\titlespacing\subsection{0pt}{0pt plus 2pt minus 2pt}{0pt plus 2pt minus 2pt}


%\renewcommand\listfigurename{圖目錄}
%\renewcommand\listtablename{表目錄} 

%%%%%%%%%%%%%%%%%%%%%%%%%%%%%

\setmainfont{Ubuntu}				%主要字型
\setmonofont{Ubuntu Mono}
\XeTeXlinebreaklocale "zh"						%中文自動換行
\XeTeXlinebreakskip = 0pt plus 1pt				%設定段落之間的距離
\setcounter{secnumdepth}{3}						%目錄顯示第三層

%%%%%%%%%%%%%%%%%%%%%%%%%%%%%
\newcommand\digitstyle{\color{DarkOrchid3}}
\makeatletter
\lst@CCPutMacro\lst@ProcessOther {"2D}{\lst@ttfamily{-{}}{-{}}}
\@empty\z@\@empty

\newtoks\BBQube@token
\newcount\BBQube@length
\def\BBQube@ResetToken{\BBQube@token{}\BBQube@length\z@}
\def\BBQube@Append#1{\advance\BBQube@length\@ne
  \BBQube@token=\expandafter{\the\BBQube@token#1}}

\def\BBQube@ProcessChar#1{%
  \ifnum\lst@mode=\lst@Pmode%
    \ifnum 9<1#1%
      \expandafter\BBQube@Append{\begingroup\digitstyle #1 \endgroup}%
    \else%
      \expandafter\BBQube@Append{#1}%
    \fi%
  \else%
    \expandafter\BBQube@Append{#1}%
  \fi%
}
\def\BBQube@ProcessStringInner#1#2\BBQube@nil{%
  \expandafter\BBQube@ProcessChar{#1}%
  \if\relax\detokenize{#2}\relax%
  \else%
    \expandafter\BBQube@ProcessStringInner#2\BBQube@nil%
  \fi%
}

\def\BBQube@ProcessString#1{\expandafter\BBQube@ProcessStringInner#1\BBQube@nil}

\lst@AddToHook{OutputOther}{%
\BBQube@ResetToken%
\expandafter\BBQube@ProcessString{\the\lst@token}%
\lst@token=\expandafter{\the\BBQube@token}%
}
\makeatother
\lstset{											% Code顯示
language=C++,										% the language of the code
basicstyle=\footnotesize\ttfamily, 						% the size of the fonts that are used for the code
%numbers=left,										% where to put the line-numbers
numberstyle=\footnotesize,						% the size of the fonts that are used for the line-numbers
stepnumber=1,										% the step between two line-numbers. If it's 1, each line  will be numbered
numbersep=5pt,										% how far the line-numbers are from the code
backgroundcolor=\color{white},					% choose the background color. You must add \usepackage{color}
showspaces=false,									% show spaces adding particular underscores
showstringspaces=false,							% underline spaces within strings
showtabs=false,									% show tabs within strings adding particular underscores
frame=false,											% adds a frame around the code
tabsize=2,											% sets default tabsize to 2 spaces
captionpos=b,										% sets the caption-position to bottom
breaklines=true,									% sets automatic line breaking
breakatwhitespace=false,							% sets if automatic breaks should only happen at whitespace
escapeinside={\%*}{*)},							% if you want to add a comment within your code
morekeywords={constexpr},									% if you want to add more keywords to the set
keywordstyle=\bfseries\color{Blue1},
commentstyle=\itshape\color{Red4},
stringstyle=\itshape\color{Green4},
}

%%%%%%%%%%%%%%%%%%%%%%%%%%%%%

\begin{document}
\pagestyle{fancy}
\fancyfoot{}
%\fancyfoot[R]{\includegraphics[width=20pt]{ironwood.jpg}}
\fancyhead[L]{National Taiwan University BestFemaleTeam}
\fancyhead[R]{\thepage}
\renewcommand{\headrulewidth}{0.4pt}
\renewcommand{\contentsname}{Contents} 

\scriptsize
\begin{multicols}{2}
  \tableofcontents
\end{multicols}
%%%%%%%%%%%%%%%%%%%%%%%%%%%%%

%\newpage

\footnotesize
\section{Basic}
\subsection{Python Test}
\lstinputlisting{Basic/test.py}
\subsection{vimrc}
\lstinputlisting{Basic/vimrc}
\subsection{Default code}
\lstinputlisting{Basic/Default_code.cpp}
\subsection{Check}
\lstinputlisting{Basic/check.sh}
\subsection{Black Magic}
\lstinputlisting{Basic/black_magic.cpp}
\subsection{C++ Random}
\lstinputlisting{Basic/random.cpp}

\section{Bitwise Trick}
\subsection{Builtin Function}
\lstinputlisting{Bitwise Trick/Builtin_function.cpp}
\subsection{Next Permutation}
\lstinputlisting{Bitwise Trick/Next_permutation.cpp}
\subsection{Subset Enumeration}
\lstinputlisting{Bitwise Trick/Subset_enumeration.cpp}

\section{STL}
\subsection{Bitset}
\lstinputlisting{STL/bitset.cpp}

\section{Data Structure}
\subsection{Discrete Trick}
\lstinputlisting{DS/Discrete_trick.cpp}
\subsection{Sparse Table}
\lstinputlisting{DS/Sparse_table.cpp}
\subsection{Treap}
\lstinputlisting{DS/Treap.cpp}
\subsection{Disjoint Set Undo ver.}
\lstinputlisting{DS/disjoint_set_undo.cpp}
\subsection{Trie}
\lstinputlisting{DS/Trie_ptr.cpp}
\subsection{Segment Tree(Range chmin, chmax, add)}
\lstinputlisting{DS/range_chmin_chmax_add_range_sum.cpp}
\subsection{2D Segment Tree}
\lstinputlisting{DS/SegTree2D.cpp}
\subsection{Li Chao Segment Tree}
\lstinputlisting{DS/li_chao.cpp}

\section{Graph}
\subsection{Bellman Ford}
\lstinputlisting{Graph/Bellman_Ford.cpp}
\subsection{SPFA}
\lstinputlisting{Graph/SPFA.cpp}
\subsection{Floyd Warshall}
\lstinputlisting{Graph/Floyd.cpp}
\subsection{Bi-CC (store vertex)}
\lstinputlisting{Graph/BCC_Vertex.cpp}
\subsection{Bi-CC (store edge)}
\lstinputlisting{Graph/BCC_edge.cpp}
\subsection{Bridge-CC}
\lstinputlisting{Graph/Bridge_CC.cpp}
\subsection{SCC}
\lstinputlisting{Graph/SCC.cpp}
\subsection{2-SAT}
\lstinputlisting{Graph/2-SAT.cpp}
\subsection{Kruskal}
\lstinputlisting{Graph/MST_Kruskal.cpp}
\subsection{Prim}
\lstinputlisting{Graph/MST_Prim.cpp}
\subsection{Euler Trail}
\lstinputlisting{Graph/Euler_Trail.cpp}
\subsection{Euler Tour}
\lstinputlisting{Graph/Euler_Tour.cpp}
\subsection{AP \& Bridge}
\lstinputlisting{Graph/Articulation_Point.cpp}
\subsection{Max Clique}
\lstinputlisting{Graph/MaxClique.cpp}
\subsection{Vizing}
\lstinputlisting{Graph/vizing.cpp}
\subsection{Dominator Tree}
\lstinputlisting{Graph/Dominator_Tree.cpp}

\section{String}
\subsection{KMP}
\lstinputlisting{String/KMP.cpp}
\subsection{String Matching bitset ver.}
\lstinputlisting{String/string_comparison_with_bitset.cpp}
\subsection{Z-value}
\lstinputlisting{String/Z_value.cpp}
\subsection{Manacher}
\lstinputlisting{String/Manacher.cpp}
\subsection{Suffix Array}
\lstinputlisting{String/SA.cpp}
\subsection{SAIS}
\lstinputlisting{String/SAIS.cpp}
\subsection{Lexicographically Smallest Rotation}
\lstinputlisting{String/Lexicographically_smallest_minimum_rotation.cpp}
\subsection{AC Automaton Arr.ver}
\lstinputlisting{String/AC_Automaton.cpp}

\section{Math}
\subsection{extgcd}
\lstinputlisting{Math/extgcd.cpp}
\subsection{Chinese Remainder Theorem}
\lstinputlisting{Math/CRT.cpp}
\subsection{Modular Multiplicative Inverse}
\lstinputlisting{Math/Modular_Inverse.cpp}
\subsection{Build Prime Table}
\lstinputlisting{Math/LinearSieve.cpp}
\subsection{Floor \& Ceil}
\lstinputlisting{Math/floor_ceil.cpp}
\subsection{Gauss Elimation Normal}
\lstinputlisting{Math/Gauss_normal.cpp}
\subsection{Gauss Elimation xor}
\lstinputlisting{Math/Gauss_xor.cpp}
\subsection{Gauss Elimation with Rank and Basis}
\lstinputlisting{Math/Gauss_withRankandBasis.cpp}
\subsection{Gaussian Integer gcd}
\lstinputlisting{Math/Gaussian_gcd.cpp}
\subsection{Fraction}
\lstinputlisting{Math/Fraction.cpp}
\subsection{Miller Rabin}
\lstinputlisting{Math/Miller_Rabin.cpp}
\subsection{Pollard Rho}
\lstinputlisting{Math/Pollard_Rho.cpp}
\subsection{Factorial without Prime Factor}
\lstinputlisting{Math/Factorial_without_prime_factor.cpp}
\subsection{Discrete Log}
\lstinputlisting{Math/Discrete_Log.cpp}
\subsection{PiCount}
\lstinputlisting{Math/PiCount.cpp}
\subsection{Möbius Function}
\lstinputlisting{Math/Mobius_Inversion.cpp}
\subsection{Sqrt under Mod}
\lstinputlisting{Math/SqrtMod.cpp}
\subsection{Sum of Floor}
\lstinputlisting{Math/SumOfFloor.cpp}

\section{Tree}
\subsection{Find Centroid}
\lstinputlisting{Tree/Centroid.cpp}
\subsection{Centroid Decomposition}
\lstinputlisting{Tree/Centroid_Decomposition.cpp}
\subsection{Heavy-Light Decomposition}
\lstinputlisting{Tree/HeavyLightDecomposition.cpp}
\subsection{LCA}
\lstinputlisting{Tree/LCA.cpp}
\subsection{Tree Hash}
\lstinputlisting{Tree/TreeHash.cpp}


\section{Geometry}
\subsection{Default Code}
\lstinputlisting{Geometry/Default_code.cpp}
\subsection{Convex Hull}
\lstinputlisting{Geometry/Convex_Hull.cpp}
\subsection{Polar Angle Sort}
\lstinputlisting{Geometry/Polar_Angle_sort.cpp}
\subsection{Intersection of two circles}
\lstinputlisting{Geometry/intersection_of_two_circles.cpp}
\subsection{Intersection of polygon and circle}
\lstinputlisting{Geometry/intersection_of_polygon_and_circle.cpp}
\subsection{Intersection of line and circle}
\lstinputlisting{Geometry/intersection_of_line_and_circle.cpp}
\subsection{PointSegDist}
\lstinputlisting{Geometry/PointSegDist.cpp}
\subsection{Rotating SweepLine}
\lstinputlisting{Geometry/rotating_sweep_line.cpp}
\subsection{Minkowski Sum}
\lstinputlisting{Geometry/Minkowski_Sum.cpp}
\subsection{Half Plane Intersection}
\lstinputlisting{Geometry/Half_Plane_Intersection.cpp}
\subsection{Polygon Area}
\lstinputlisting{Geometry/polyArea.cpp}
\subsection{Polygon Union Area}
\lstinputlisting{Geometry/polyUnion.cpp}

\section{Flow}
\subsection{SW\_MinCut}
\lstinputlisting{Flow/SW_MinCut.cpp}
\subsection{Kuhn Munkres}
\lstinputlisting{Flow/Kuhn_Munkres.cpp}
\subsection{Bipartite Graph Matching}
\lstinputlisting{Graph/Bipartite_Matching.cpp}
\subsection{General Graph Matching}
\lstinputlisting{Flow/General_Graph_Matching.cpp}
\subsection{Maximum Simple Graph Matching}
\lstinputlisting{Flow/Maximum_Simple_Graph_Matching.cpp}
\subsection{Minimum Weight Matching (Clique version)}
\lstinputlisting{Flow/Minimum_Weight_Matching.cpp}
\subsection{Dinic}
\lstinputlisting{Flow/Dinic.cpp}

\section{Convolution}
\subsection{FFT}
\lstinputlisting{Convolution/FFT.cpp}
\subsection{NTT}
\lstinputlisting{Convolution/NTT.cpp}
\subsection{FWT}
\lstinputlisting{Convolution/FWT.cpp}

\section{Else}
\subsection{Second-Best Minimum Spanning Tree}
\lstinputlisting{Else/SecondBestMST.cpp}
\subsection{Algorithm Note}
\lstinputlisting{Else/note.cpp}
\section{Python}
\subsection{Misc}
\lstinputlisting{Python/misc.py}


\end{document}
